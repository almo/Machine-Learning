\documentclass[a4paper, 11pt, bibliography=numbered, es]{article}
\usepackage[spanish]{babel}
\usepackage{graphicx} % Required for inserting images
\usepackage{multirow} % Required for tables
\usepackage{graphicx} %Require for images/graph
\usepackage{amsmath}
\usepackage[utf8]{inputenc}
\usepackage{hyperref}
\usepackage{subfigure}
\usepackage{longtable}
\usepackage[backend=biber,style=numeric]{biblatex}
\usepackage{csquotes}
\usepackage{parskip}

\setlength\parindent{0pt}
\addbibresource{almo.bib}

\title{Alfabetización \& Estrés Financiero}
\author{Andrés-Leonardo Martínez-Ortiz \\ amartinez122@alumno.uned.es}
\date{Diciembre 2024}

\begin{document}
\maketitle
\section{Introducción}
En la actualidad, los individuos de las sociedad desarrolladas, cuentan con 
un sinfín de mecanismos financieros, que dan soporte a la vida contemporánea
y constituyen herramientas catalizadoras de proyectos personales y empresariales. La adecuada integración, el aprovechamiento y el desarrollo de
nuestra sociedad, sin duda, requieren del conocimiento financiero básico, así como
de los servicios y herramientas ofrecidas. Su desconocimiento, no solo puede limitar
nuestro desarrollo y participación en la sociedad, sino que en los casos más extremos
puede impactar negativamente, con implicaciones que se prolonguen a lo largo de nuestra
vida. 

La alfabetización financiera, baremo elemental de educación en este dominio, 
proporciona el conocimiento y habilidades para tomar decisiones como la 
apertura de una cuenta bancaria, cómo obtener recursos para comprar una
casa o financiar unos estudios, dónde invertir nuestros ahorros y cómo 
garantizar nuestra estabilidad financiera a lo largo de nuestra vida
\cite{EU01}, \cite{OCDE01}. Su importancia más allá de lo estrictamente personal vendría respaldad por algunos estudios que aportan evidencia de
la conexión entre alfabetización digital y el desarrollo económico \cite{Lusardi14}. 

\section{Motivación y objetivos del trabajo}
\label{motivacion}
El presente trabajo aborda el análisis de los aspectos que caracterizan el conocimiento financiero y las consecuencias, englobadas bajo el término de estrés financiero\footnote{El término estrés financiero se definirá como parte del presente trabajo.}, que se derivan de una insuficiente alfabetización en esta materia. Su ámbito se limita a datos de
sección cruzada, obtenidos en 2021 y referidos a población estadounidense \cite{NFCS01}.

El objetivo es poder caracterizar la alfabetización financiera en términos demográficos e información individual básica de uso de servicios y aplicaciones, introduciendo una taxonomía de niveles de estrés financiero, que permita la definición de medidas no solo 
educativas de prevención, sino también acciones reactivas de carácter paliativo.

Como pasos futuros, aunque fuera ya del ámbito de este trabajo, la expansión del contexto geográfico y el análisis de datos de panel, permitiría entender el impacto de las medidas 
adoptadas, no ya en los aspectos abordados en este primer estudio, sino también en otros
aspectos como el desarrollo económico, la iniciativa empresarial y la inversión privada. 
\section{Descripción y justificación de la técnica o técnicas utilizadas}
Los objetivos establecidos en el apartado \ref{motivacion} se alcanzarán 
aplicando técnicas de aprendizaje estadístico, como aproximación alternativa a los modelos econométricos convencionales. 

En una primera fase y siguiendo un proceso de aprendizaje no supervisado, se definirá la taxonomía que permita clasificar el espacio social de estudio con criterios de estrés financiero. A continuación, y una vez etiquetados los individuos de la muestra, se 
desarrollarán clasificadores estadísticos que permitan predecir la categoría de un 
individuo y las correspondientes medidas de prevención y reacción. 

Para el análisis del problema y desarrollo de los modelos propuestos, se seguirán las 
técnicas correspondientes a máquinas de vector soporte y árboles de decisión, descritas en las referencias \cite{lantz23}, \cite{Hastie23} y \cite{Hastie13}, y programadas en R\cite{R24}.
%%\section{Desarrollo de la investigación}
%%\section{Análisis y evaluación de los resultados obtenidos}
%%\section{Conclusiones}
%%\section{Anexo: código}
\printbibliography 
\end{document}
