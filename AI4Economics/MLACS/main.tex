\documentclass[a4paper, 11pt, bibliography=numbered, es]{article}
\usepackage[spanish]{babel}
\usepackage{graphicx} % Required for inserting images
\usepackage{multirow} % Required for tables
\usepackage{graphicx} %Require for images/graph
\usepackage{amsmath}
\usepackage[utf8]{inputenc}
\usepackage{hyperref}
\usepackage{subfigure}
\usepackage{longtable}
\usepackage[backend=biber,style=numeric]{biblatex}
\usepackage{csquotes}
\usepackage{parskip}

\setlength\parindent{0pt}
\addbibresource{almo.bib}

\title{Alfabetización \& Estrés Financiero}
\author{Andrés-Leonardo Martínez-Ortiz \\ amartinez122@alumno.uned.es}
\date{Diciembre 2024}

\begin{document}
\maketitle
\section{Introducción}
En la actualidad, los individuos de las sociedad desarrolladas, cuentan con 
un sinfín de mecanismos financieros, que dan soporte a la vida contemporánea
y constituyen herramientas catalizadoras de proyectos personales y empresariales. 
La adecuada integración, el aprovechamiento y el desarrollo de
nuestra sociedad, sin duda, requieren del conocimiento financiero básico, así como
de los servicios y herramientas ofrecidas. Su desconocimiento, no solo puede limitar
nuestro desarrollo y participación en la sociedad, sino que en los casos más extremos
puede impactar negativamente, con efectos que se prolongan a lo largo de nuestra
vida. 

La alfabetización financiera, baremo elemental de educación en este dominio, 
proporciona el conocimiento y habilidades para tomar decisiones como la 
apertura de una cuenta bancaria, cómo obtener recursos para comprar una
casa o financiar unos estudios, dónde invertir nuestros ahorros y cómo 
garantizar nuestra estabilidad financiera a lo largo de nuestra vida
\cite{EU01}, \cite{OCDE01}. Su importancia más allá de lo estrictamente personal 
vendría respaldada por algunos estudios que aportan evidencia de
la conexión entre alfabetización digital y el desarrollo económico \cite{Lusardi14}. 

\section{Motivación y objetivos del trabajo}
\label{motivacion}
El presente trabajo aborda el análisis de los aspectos que caracterizan el conocimiento 
financiero y las consecuencias, englobadas bajo el término de estrés financiero\footnote{El término
estrés financiero se definirá como parte del presente trabajo.}, que se derivan de una insuficiente
alfabetización en esta materia. Su ámbito se limita a datos de sección cruzada, obtenidos en 2021 y 
referidos a población estadounidense \cite{NFCS01}.

El objetivo es poder caracterizar la alfabetización financiera en términos demográficos e información
individual básica de uso de servicios y aplicaciones, introduciendo una taxonomía de niveles de estrés
financiero, que permita la definición de medidas no solo educativas de prevención, sino también acciones
reactivas de carácter paliativo.

Como pasos futuros, aunque fuera ya del ámbito de este trabajo, la expansión del contexto geográfico
y el análisis de datos de panel, permitiría entender el impacto de las medidas adoptadas, no ya en los
aspectos abordados en este primer estudio, sino también en otros aspectos como el desarrollo económico,
la iniciativa empresarial y la inversión privada. 

\section{Descripción y justificación de la técnica o técnicas utilizadas}
Los objetivos establecidos en el apartado \ref{motivacion} se alcanzarán aplicando técnicas de aprendizaje
estadístico, como aproximación alternativa a los modelos econométricos convencionales. 

En una primera fase y siguiendo un proceso de aprendizaje no supervisado, se definirá la taxonomía que permita
clasificar el espacio social de estudio con criterios de estrés financiero. A continuación, y una vez etiquetados
los individuos de la muestra, se desarrollarán clasificadores estadísticos que permitan predecir la categoría de un 
individuo, permitiendo anticipar las correspondientes medidas de prevención y dirigiendo las acciones de reacción. 

Para el análisis del problema y desarrollo de los modelos propuestos, se seguirán las técnicas correspondientes
a algoritmos de clasificación no supervisada, máquinas de vector soporte y árboles de decisión, descritas en las
referencias \cite{lantz23}, \cite{Hastie23}, \cite{aurelien17} y \cite{Hastie13}, y programadas en R\cite{R24}.

El desarrollo de la investigación sigue una secuencia de pasos: (1) selección y descripción de la fuente de datos utilizada,
(2) la selección de los métodos de análisis, (3) la preparación de los datos para su proceso y finalmente (4) la selección
de los parámetros y entrenamiento de los métodos supervisados. Una vez completada esta parte, la aplicación de
las técnicas obtenidas permitirá el análisis y evaluación de los resultados, lo que permitirá alcanzar un conjunto
de conclusiones.

\section{Desarrollo de la investigación}
\subsection{Datos: selección y descripción}
La fuente de datos seleccionada corresponde al proyecto National Finantial Capability Study \cite{NFCS01}, 
un estudio a gran escala sobre la capacidad financiera de los ciudadanos estadounidenses. La muestra se 
compone de las respuestas a un cuestionario de 27118 adultos, con una distribución aproximada de 500
individuos por estado, más el Distrito de Columbia. Adicionalmente, requerimientos del estudio han obligado
a aumentar el tamaño de la muestra en los estados de California y de Oregon hasta alcanzar los 1250 individuos.
A nivel de estado se establecieron cuotas demográficas para garantizar la representatividad a nivel de edad,
genero, educación e ingresos.

El cuestionario utilizado para recavar los datos contiene 126 variables explicativas, identificadas con códigos 
alfanuméricos (J10, J41\_1, etc \dots) y que son de tipo nominal, representadas numéricamente. Tanto la estructura
del conjunto de datos, como la asignación de valores numéricos y etiquetas, se encuentran descrito en el fichero 
\textit{NFCS 2021 State Data File Info 220627} que se distribuye con los datos\footnote{Recursos disponibles online https://finrafoundation.org/data-and-downloads}. 

En esta investigación no se harán uso de todas las variables explicativas disponibles, reduciendo el análisis a un total de 51 variables. Serán excluidas aquellas sin relación, que presentan cierto nivel de redundancia o que no aportan valor según lo
objetivos establecidos. Además se han organizado en tres grupos, recogiendo información demográfica, de estrés financiero y
capacitación financiera. A continuación se presentan el conjunto de datos con la estructura propuesta.

Un total de 14 variables explicativas caracterizan la \textbf{información demográfica} del individuo:
\begin{enumerate}
    \item \textbf{STATEQ} State
    \item \textbf{A50A} GENDER (non-binary randomly assigned)
    \item \textbf{A3Ar\_w} Age group
    \item \textbf{A50B} GENDER/AGE NET (non-binary randomly assigned)
    \item \textbf{A5\_2015} What was the highest level of education that you completed?
    \item \textbf{A6} What is your marital status?
    \item \textbf{A7} Which of the following describes your current living arrangements?
    \item \textbf{A11} How many children do you have who are financially dependent on you (or your spouse/partner)? 
    \item \textbf{A8\_2021} What is your (household's) approximate annual income, including wages, tips, investment
    income, public assistance, income from retirement plans, etc.?
    \item \textbf{A9} Which of the following best describes your current employment or work status?
    \item \textbf{A40} In addition to your main employment, did you also do other/Did you do any work for pay
    in the past 12 months?
    \item \textbf{A10} Which of the following best describes your [spouse's/partner's] current employment or
    work status? 
    \item \textbf{A41} What was the highest level of education completed by the person or any of the people who
    raised you?
    \item \textbf{A21\_2015} Are you a part-time student taking courses for credit?
\end{enumerate}

La \textbf{alfabetización financiera} vendrá caracterizada por las siguientes 16 variables explicativas:
\begin{enumerate}
    \item \textbf{B1} Do you (Does your household) have a checking account? 
    \item \textbf{B2} Do you [Does your household] have a savings account, money market account, or CDs?
    \item \textbf{B31} How often do you use your mobile phone to pay for a product or service in person at a store,
    gas station, or restaurant (e.g., by waving/tapping your mobile phone over a sensor at checkout, scanning
    a barcode or QR code using your mobile phone, or using so 
    \item \textbf{B42} How often do you use your mobile phone to transfer money to another person?
    \item \textbf{B43} How often do you use websites or apps to help with financial tasks such as budgeting, saving,
    or credit management (e.g., GoodBudget, Mint, Credit Karma, etc.)? Please do not include websites or 
    apps for making payments or money transfers.
    \item \textbf{C1\_2012} Do you (or your spouse/partner) have any retirement plans through a current or previous 
    employer, like a pension plan, (a Thrift Savings Plan (TSP),) or a 401(k)?
    \item \textbf{C2\_2012} Were these plans provided by your employer or your (spouse's/partner's) employer, or both? 
    \item \textbf{C5\_2012} Do you [or your spouse/partner] regularly contribute to a retirement account like a (Thrift
    Savings Plan (TSP),) 401(k) or IRA?   
    \item \textbf{B14} Not including retirement accounts, do you  (does your household) have any investments in stocks, bonds, mutual funds, or other securities? 
    \item \textbf{EA\_1}  Do you (or your spouse/partner) currently own your home?
    \item \textbf{E7} Do you currently have any mortgages on your home?
    \item \textbf{F1} How many credit cards do you have?  
    \item \textbf{H1}  Are you covered by health insurance?
    \item \textbf{M1\_1} How strongly do you agree or disagree with the following statements? - I am good at dealing with day-to-day financial matters, such as checking accounts, credit and debit cards, and tracking expenses
    \item \textbf{M4} On a scale from 1 to 7, where 1 means very low and 7 means very high, how would you assess your overall financial knowledge?
    \item \textbf{M20} Was financial education offered by a school or college you attended, or a workplace where you were employed?
\end{enumerate}

Por ultimo, el \textbf{estrés financiero} se caracteriza con un las siguientes 23 variables explicativas:
\begin{enumerate}
    \item \textbf{J1} Overall, thinking of your assets, debts and savings, how satisfied are you with your current
    personal financial condition? 
    \item \textbf{J3} Over the past year, would you say your (household's) spending was less than, more than, or about
    equal to your (household's) income?
    \item \textbf{J4} In a typical month, how difficult is it for you to cover your expenses and pay all your bills?
    \item \textbf{J5} Have you set aside emergency or rainy day funds that would cover your expenses for 3 months,
    in case of sickness, job loss, economic downturn, or other emergencies?
    \item \textbf{J6} Are you setting aside any money for your children's college education?
    \item \textbf{J10} In the past 12 months, have you [has your household] experienced a large  drop in income which
    you did not expect? 
    \item \textbf{J20} How confident are you that you could come up with \$2,000 if an unexpected need arose within
    the next month?
    \item \textbf{J32} How would you rate your current credit record?
    \item \textbf{C10\_2012} In the last 12 months, have you (or your spouse/partner) taken a loan from your retirement account(s)? 
    \item \textbf{E15\_2015} How many times have you been late with your mortgage payments in the past 12 months?
    \item \textbf{F2\_1}  In the past 12 months, which of the following describes your experience with credit cards? - I always paid my credit cards in full
    \item \textbf{F2\_2} In the past 12 months, which of the following describes your experience with credit cards? - In some months, I carried over a balance and was charged interest
    \item \textbf{F2\_3} In the past 12 months, which of the following describes your experience with credit cards? - In some months, I paid the minimum payment only
    \item \textbf{P50} At any time in your adult life (18 and older), did your parents or grandparents pay for an expense of yours that was \$10,000 or more?
    \item \textbf{F2\_4} In the past 12 months, which of the following describes your experience with credit cards? - In some months, I was charged a late fee for late payment 
    \item \textbf{F2\_5} In the past 12 months, which of the following describes your experience with credit cards? - In some months, I was charged an over the limit fee for exceeding my credit line 
    \item \textbf{F2\_6} In the past 12 months, which of the following describes your experience with credit cards? - In some months, I used the cards for a cash advance 
    \item \textbf{G20} Do you currently have any unpaid bills from a health care or medical service provider (e.g., a hospital,
     a doctor's office, or a testing lab) that are past due?
    \item \textbf{G35} How many times have you been late with a student loan payment in the past 12 months?
    \item \textbf{G38} Have you been contacted by a debt collection agency in the past 12 months?
    \item \textbf{H30\_1} In the last 12 months, was there any time when you… - Did NOT fill a prescription for medicine because of the cost
    \item \textbf{H30\_2} In the last 12 months, was there any time when you… - SKIPPED a medical test, treatment or follow-up recommended by a doctor because of the cost
    \item \textbf{H30\_3} In the last 12 months, was there any time when you… - Had a medical problem but DID NOT go to a doctor or clinic because of the cost
\end{enumerate}

Adicionalmente se considera el campo NFCSID que es un indicador de cada registro
del conjunto de datos. Esto permitirá conectar los distintos resultados.

\subsection{Análisis exploratorio de los Datos}
%%\section{Análisis y evaluación de los resultados obtenidos}
%%\section{Conclusiones}
%%\section{Anexo: código}
\printbibliography 
\end{document}
